\documentclass[11pt,reqno]{amsart}
%\newlength\tindent
%\setlength{\tindent}{\parindent}
%\setlength{\parindent}{0pt}
%\renewcommand{\indent}{\hspace*{\tindent} \bigskip}

\usepackage{setspace}
\usepackage[margin=1in]{geometry}
\usepackage[hang,flushmargin,symbol*]{footmisc}
\usepackage{amsmath}
\usepackage{amsthm}
\usepackage{amssymb}
\usepackage{mathtools}
\usepackage{enumitem}
\usepackage{calc}
\usepackage{graphicx}
\usepackage{caption}
\usepackage[labelformat=simple,labelfont={}]{subcaption}
\usepackage{tikz}
\usetikzlibrary{decorations.markings}
\usetikzlibrary{arrows,shapes,positioning}
\usetikzlibrary{patterns}
\usepackage{url}
\usepackage{array}
\usepackage{graphicx}
\usepackage{color}
\usepackage{mathrsfs}

\usepackage{verbatim}

\usepackage{subfiles}


\usepackage[colorinlistoftodos, loadshadowlibrary]{todonotes}
%\usepackage[disable]{todonotes}

\setlength{\marginparwidth}{.8in}

\usepackage{dashbox}
\newcommand\dboxed[1]{\dbox{\ensuremath{#1}}}

\theoremstyle{definition}
\newtheorem{theorem}{Theorem}[section]
\newtheorem{lemma}[theorem]{Lemma}
\newtheorem{claim}[theorem]{Claim}
\newtheorem{corollary}[theorem]{Corollary}
\newtheorem{proposition}[theorem]{Proposition}
\newtheorem{conjecture}[theorem]{Conjecture}
\newtheorem{definition}[theorem]{Definition}
\newtheorem{example}[theorem]{Example}
\newtheorem{remark}[theorem]{Remark}
\newtheorem{question}[theorem]{Question}
\newtheorem{answer}[theorem]{Answer}
\newtheorem{construction}[theorem]{Construction}
\newtheorem{situation}[theorem]{Situation}
\newtheorem{assumption}[theorem]{Assumption}
\newtheorem{desiderata}[theorem]{Desiderata}
\newtheorem{warning}[theorem]{Warning}






%\newtheorem{case}[theorem]{Case}


\usepackage{amsfonts}
\usepackage{amsmath}
\usepackage{amssymb}
\usepackage{mathrsfs}
\usepackage{tikz}
\usepackage{tikz-cd}
\usepackage{hyperref}
\usepackage{ mathdots }
\usepackage{dutchcal}
\usepackage{verbatim}
\usepackage{amsmath,amsthm}
\usepackage{extarrows}

%Lets me use if then statements and create separate moduli of curves commands for if there is or is no input
\usepackage{xparse}



\newcommand{\BFjet}[1]{J^{BF}_{#1}}



%Comment out this package to make labels invisible
\usepackage{showlabels}

\usepackage{multirow}
\usepackage{multicol}

%strikethrough text
\usepackage[normalem]{ulem}
%strikethrough underlines in math mode unless you do this stupid modification:



\newcommand{\scr}[1]{\ensuremath{\mathscr{#1}}}














\newcommand{\msout}[1]{\text{\sout{\ensuremath{#1}}}}


\newcommand{\ZZ}{\mathbb{Z}}
\newcommand{\CC}{\mathbb{C}}
\newcommand{\NN}{{\mathbb{N}}}
\newcommand{\QQ}{{\mathbb{Q}}}
\newcommand{\OO}{\mathcal{O}}
\newcommand{\MM}{{M}}
\newcommand{\RR}{\mathbb{R}}
\newcommand{\Aff}{{\mathbb{A}}}
\newcommand{\PP}{\mathbb{P}}

\newcommand{\GG}{\mathbb{G}}

\newcommand{\Gtrop}{\GG_{trop}}
\newcommand{\Glog}{\GG_{log}}

\newcommand{\fanskel}[1]{{\text{Sk}^{#1}}}

\newcommand{\nn}[1]{{\mathscr{N}_{#1}}}
\newcommand{\N}[1]{{N_{#1}}}
\newcommand{\Nl}[1]{{N_{#1}^{\ell}}}
\newcommand{\Nlstrict}[1]{{N_{#1}^{\msout{\ell}}}}

\newcommand{\Div}{{\text{Div}}}


\newcommand{\inc}{\rm inc}
\newcommand{\ft}{\rm ft}

\newcommand{\lHD}{{\rm HD}^\ell}





\newcommand{\Spec}{{\rm{Spec}\:}}
\newcommand{\SSpec}[1]{{\underline{\rm{Spec}}_{#1}\:}}
\newcommand{\Proj}{{{\rmProj}\:}}
\newcommand{\PProj}{\underline{\text{Proj}}}
\newcommand{\Hom}{{\rm Hom}}
\newcommand{\HHom}{\underline{{\rm Hom}}}
\newcommand{\lHHom}{\underline{\text{Hom}}^\ell}
\newcommand{\Isom}{\text{Isom}}
\newcommand{\IIsom}{\underline{\text{Isom}}}
\newcommand{\EExt}{\underline{\text{Ext}}}
\newcommand{\TTor}{\text{Tor}}
\newcommand{\Der}{\text{Der}}
\newcommand{\Exal}{\text{Exal}}
\newcommand{\Ext}{\text{Ext}}
\newcommand{\Mext}{\text{MExt}}
\newcommand{\hh}[1]{\mathcal{h}_{#1}}
\newcommand{\vH}{\check{H}}
\newcommand{\inpt}{{\underline{\phantom{ab}}}}
\newcommand{\ccx}[1]{{\mathbb{L}_{#1}}}
\newcommand{\ccxd}[1]{{\mathbb{L}^\Delta_{#1}}}
\newcommand{\lccx}[1]{{\mathbb{L}^\ell_{#1}}}
\newcommand{\lgabcx}[1]{{\mathbb{L}^G_{#1}}}

\newcommand{\Con}[1]{\text{Const}_{#1}}

\newcommand{\gp}[1]{#1^{\rm gp}}
\newcommand{\vfc}[2]{ {[{#1}, {#2}]^{vir}} }
\newcommand{\lvfc}[2]{ {[{#1}, {#2}]^{\ell vir}} }

\newcommand{\glob}{\mathcal{V}}
\newcommand{\rectglob}{\mathcal{W}}



\newcommand{\ev}{\hat \wedge}



\def\lcm{\operatorname{lcm}}

\newcommand{\Cl}[1]{{C_{#1}^\ell}}
\newcommand{\Clstrict}[1]{{C_{#1}^{\msout{\ell}}}}
\newcommand{\C}[1]{C_{#1}}

\newcommand{\DNC}[1]{{\widetilde{M}_{#1}}}

\newcommand{\Tl}[1]{{T^{\ell}_{#1}}}
\newcommand{\lkah}[1]{\Omega^\ell_{#1}}
\newcommand{\kah}[1]{\Omega_{#1}}
\newcommand{\Log}{{\mathcal{L}og}}
\newcommand{\Tor}{{\mathscr{T}}}
\newcommand{\lpb}{{\arrow[dr, phantom, very near start, "\ulcorner \ell"]}}
\newcommand{\lpbstrict}{{\arrow[dr, phantom, very near start, "\ulcorner \msout{\ell}"]}}
\newcommand{\pb}{{\arrow[dr, phantom, very near start, "\ulcorner"]}}

\newcommand{\num}[1]{\langle #1 \rangle}

%fixing tilde and hat
\renewcommand{\tilde}[1]{\widetilde{#1}}
\renewcommand{\hat}[1]{\widehat{#1}}


\newcommand{\stquot}[1]{\left[ #1 \right]}


\newcommand{\lct}{{\rm lct}}




\def\overnorm#1{\overline{#1}\vphantom{#1}}

\renewcommand{\bar}[1]{\ensuremath{\overnorm{#1}}}

\newcommand{\Ms}{{\overline{M}_{g, n}}}
\newcommand{\Msi}[1]{{\overline{M}_{g, n} (#1)}}
\newcommand{\Msg}[1]{{\overline{M}_{\Gamma} (#1)}}
%to make $M_{g, n+m}$ look good
\newcommand{\Msp}[1]{{\overline{M}_{g, n+#1}}}
\newcommand{\Mspecific}[1]{\overline{M}_{#1}}

\newcommand{\Ml}{{\mathscr{M}^\ell_{g, n}}}
\newcommand{\Mli}[1]{{\mathscr{M}^\ell_{g, n} (#1)}}
\newcommand{\Mlg}[1]{{\mathscr{M}^\ell_{\Gamma} (#1)}}


\newcommand{\point}{{\overline{o}}}



\newcommand{\Mprel}{{\mathfrak{M}_{g, n}}}


\newcommand{\Aut}{\underline{\text{Aut}}}


\newcommand{\Tc}{{\mathscr{T}}}



\newcommand{\val}[1]{{#1}^{val}}
\newcommand{\rval}[1]{{#1}^{\infty val}}

\newcommand{\CDiv}[1]{\text{CDiv}(#1)}
\newcommand{\WDiv}[1]{\text{WDiv}(#1)}

\newcommand{\lCDiv}[1]{\text{CDiv}^\ell(#1)}
\newcommand{\lWDiv}[1]{\text{WDiv}^\ell(#1)}


\def\Cone{\operatorname{Cone}}

\newcommand{\WP}{\cal{WP}}
\newcommand{\WE}{\cal{WE}}



\newcommand{\UU}{{\mathcal{U}}}
\newcommand{\UUU}{{\widetilde{\mathcal{U}}}}

\newcommand{\Sym}{\text{Sym}\,}

\newcommand{\DS}[1]{P_{#1}}


\newcommand{\longequals}{\xlongequal{\: \:}}
\newcommand{\coker}{\text{coker}}
\newcommand{\action}{\:\rotatebox[origin=c]{-90}{$\circlearrowright$}\:}
\newcommand{\Split}{\text{Split}}
\newcommand{\colim}{\text{colim}}
\newcommand{\et}{{\text{\'et}}}

%commands in case jonathan doesn't like my terminology
\newcommand{\lpot}{{Log POT}\,}
\newcommand{\lvirt}{{Log VFC}\,}


%brackets
\usepackage{stmaryrd}
\newcommand{\adj}[1]{\llbracket #1 \rrbracket}



%jets and log jets
\newcommand{\ljet}[2]{\Delta_{#2}^{(#1)}}
\newcommand{\tljet}[2]{\scr D_{#2}^{(#1)}}
\newcommand{\jet}[1]{\Delta_{#1}}

%(log) Jet Spaces
\newcommand{\ljsp}[2]{J^{(#1)}_{#2}}
\newcommand{\tljsp}[2]{\scr J^{(#1)}_{#2}}
\newcommand{\ljspnonfs}[2]{S^{(#1)}_{#2}}
\newcommand{\jsp}[1]{J_{#1}}
\newcommand{\tjsp}[1]{\scr J_{#1}}

\newcommand{\FF}{\mathbb F}

\newcommand{\usch}[1]{{#1}^\circ}

\newcommand{\WR}{\cal R}
\newcommand{\WRc}{\hat{\cal R}}
\newcommand{\Gen}{\cal M}



\newcommand{\smet}{\Acute{e}t}

\newcommand{\bra}[1]{\left[{#1}\right]}



\usepackage{bm}
\usepackage{xspace}
%Names for different log structures on log jets
%don't use this -- it's deprecated. Just use DIV with a number
\newcommand{\BF}{\ensuremath{\boldsymbol{BF}}\xspace}
\newcommand{\HOL}{\ensuremath{\boldsymbol{HOL}}\xspace}
\newcommand{\DIV}[1]{\ensuremath{\boldsymbol{DIV}_{#1}}\xspace}


\newcommand{\AF}[1][{}]{\Theta_{{#1}}}
\newcommand{\af}[1][{}]{\AF[#1]}



\newcommand{\cal}[1]{\ensuremath{\mathcal{#1}}}


\newcommand{\pt}{\Spec k}

\newcommand{\VV}{\mathbb{V}}

\newcommand{\bb}[1]{\ensuremath{\mathbb{#1}}}

\newcommand{\rk}{{\rm rk}}


\newcommand{\dobib}{
\onlyinsubfile{
\bibliographystyle{alpha}%Used BibTeX style is unsrt
\bibliography{zbib}}}




\newcommand{\klvar}[1]{K^{log}_0({\rm Var}_{#1})}
\newcommand{\ktropvar}[1]{K^{trop}_0({\rm Var}_{#1})}
\newcommand{\otherklvar}[1]{K_0({\rm Log}_{#1})}


\newcommand{\klbitt}[1]{K^{log Bitt}_0({\rm Var}_{#1})}


\newcommand{\ordkvar}[1]{K_0({\rm Var}_{#1})}


\newcommand{\LL}{\mathbb{L}}

\newcommand{\Cones}{(Cones)}
\newcommand{\Mons}{(Mon)}
\newcommand{\longsimeq}{\overset{\sim}{\longrightarrow}}

\newcommand{\HS}{{\rm HS}}

\newcommand{\jl}[2]{#1_{#2}^\ell}



\newcommand{\Leo}[2][inline]{\todo[linecolor=blue,backgroundcolor=blue!25,bordercolor=blue,#1,shadow,author=Leo]{#2}} %Todo notes for LEO. 

\newcommand{\TdF}[2][inline]{\todo[linecolor=red,backgroundcolor=red!25,bordercolor=red,#1,shadow,author=Tommaso]{#2}} %Todo notes for Tommaso. 

\newcommand{\End}{{\rm End}}

\newcommand{\KN}[1]{#1^{KN}}


\newcommand{\lmu}{\mu^\ell}

\title{Title}
\author{Everybody}
\date{\today}

\begin{document}

\maketitle

\section{Introduction}


Let $R$ be a ring, and write $R_{\ZZ}$ for its underlying abelian group. Each $r \in R$ induces an endomorphism
\[\cdot r : R_\ZZ \to R_\ZZ; \qquad s \mapsto sr\]
by multiplication. This gives a ring map
\begin{equation}\label{eqn:psiringabelianZ}
	\Psi : R \to \End(R_\ZZ),
\end{equation}
where $\End(R_\ZZ) \coloneqq \Hom_{\rm ab}(R_\ZZ, R_\ZZ)$ is the ring of endomorphisms of $R_\ZZ$ as an abelian group. The map $\Psi$ is always injective, as the image of $1 \in R_\ZZ$ distinguishes elements of $R$. 


\begin{question}[H. Lenstra]\label{q:lenstra}

When is $\Psi$ also surjective?

\end{question}


Such rings are present in the literature as ``E-rings'' \url{https://arxiv.org/pdf/math/0404271.pdf}. 

We generalize the map $\Psi$.

Let $S \to R$ be a map of rings and write $R_S$ for $R$ regarded as an $S$-module. We similarly obtain a map
\begin{equation}\label{eqn:ringabelian}
\Psi : R \to \End(R_S); \qquad r \mapsto \cdot r.
\end{equation}
Endomorphisms are $S$-module maps $\Hom_S(R, R)$, which are equated with $R$-module maps
\[\Hom_S(R, R) \simeq \Hom_R(R \otimes_S R, R)\]
via tensor-hom adjunction. The map
\[\Psi : R \to \End(R_S)\]
is dual to the multiplication map $R \otimes_S R \to R$:
\[\Psi : R \simeq \Hom_R(R, R) \to \Hom_R(R \otimes_S R, R).\]

\begin{remark}

If $S = \ZZ$, this reproduces the $\Psi$ above. If $S \to R$ is an epimorphism, $R \otimes_S R = R$ and $\Psi$ is an isomorphism. 


\end{remark}



Write $I \coloneqq \ker(R \otimes_S R \to R)$ for the kernel of the multiplication map. Fix a morphism $\varphi : R \otimes_S R \to R$ in $\Hom_R(R \otimes_S R, R)$. The map $\varphi$ comes from a map $R \to R$ if and only if the restriction to the ideal is zero:
\[
\begin{tikzcd}
0 \ar[r] 	&I \ar[dr] \ar[r] 		&R \otimes_S R \ar[d, "\varphi"] \ar[r] 		&R \ar[r] \ar[dl, dashed, "\exists"] 		&0 		\\
		&&R
\end{tikzcd}
\]

We give a tautological answer to Question \ref{q:lenstra} before elaborating. 

\begin{answer}

The map $\Psi$ of \eqref{eqn:ringabelian} is surjective if and only if, for all maps $\varphi : R \otimes_S R \to R$, the restriction $\varphi|_I$ is zero. 

\end{answer}





The map $R \otimes_S R \to R$ has kernel $I$ generated by 
The kernel $I$ of the map $R \otimes_S R \to R$ is generated by 
\[(r_1 r_2 \otimes r_3 - r_1 \otimes r_2 r_3 \, | \, r_1, r_2, r_3 \in R) \quad \subseteq R \otimes_S R.\] 
This is because modding out $R \otimes_S R$ by such a relation yields $R \otimes_R R = R$. There is a surjection
\[R^{\otimes_s^3} \to I; \qquad r_1 \otimes r_2 \otimes r_3 \mapsto r_1 r_2 \otimes r_3 - r_1 \otimes r_2 r_3.\]
This is the differential in the normalized bar resolution. A map $\varphi : R \otimes_S R \to R$ comes from a map $f : R \to R$ if and only if 
\begin{equation}\label{eqn:tensorprodrelns}
\varphi(r_1r_2, r_3) = \varphi(r_1, r_2 r_3), \qquad r_1, r_2, r_3 \in R.
\end{equation}



By \cite[0BSZ]{stacks-project}, the dualizing sheaf of the diagonal $R \otimes_S R \to R$ is the hom-set
\Leo{This is otherwise known as $T_{R/S}$!}
\[\Hom_R(R \otimes_S R, R) = \omega_{R/R \otimes_S R}.\]
This dualizing sheaf has a canonical section $\tau \in \omega$ given by the multiplication map. The map $\Psi$ of \eqref{} is surjective if and only if $\tau \in \omega$ generates $\omega$ as an $R$-module. 

There is a universal open locus in $\Spec R$ on which $\tau \in \omega$ generates it as an $R$-module: let $D \subseteq \Spec R$ be the support of the cokernel of 
\[\OO_{\Spec R} \to \omega; \qquad 1 \mapsto \tau.\]
The subscheme $D$ is closed, as the cokernel is a coherent sheaf, and the open locus is just the complement $\Spec R \setminus D$. 

One can further stratify ``how far away'' $R$ is from being an $E$-ring over $S$ by taking the Fitting-ideal locally closed stratification on which $\omega$ needs a certain number of generators:
\[X = \Spec R = X_0 \sqcup X_1 \sqcup X_2 \cdots .\]

The element $\tau \in \omega$ only has a hope of generating if $\omega$ locally can be generated by $\leq 1$ element, so the $E$-ring locus is contained in $X_0 \sqcup X_1$. 

\begin{remark}
	
This is analogous to the trace pairing being nondegenerate precisely when an extension or order is \'etale. In that case, the trace pairing is another canonical element $\tau \in \omega$ in the dualizing sheaf of the order, and it vanishes along the discriminant. This is only an analogy, as no such trace pairing can be constructed for the diagonal. 

\end{remark}


We can express the condition \eqref{eqn:tensorprodrelns} in these terms. Write 
\[\omega_2 \coloneqq \Hom_R(R \otimes_S R, R), \qquad \omega_3 \coloneqq \Hom_R(R^{\otimes_S 3}, R)\]
for the dualizing sheaves of the diagonal and triple diagonal. Pulling back along the differential
\[d : R^{\otimes_S 3} \to R^{\otimes_S 2}; \qquad r_1 \otimes r_2 \otimes r_3 \mapsto r_1 r_2 \otimes r_3 - r_1 \otimes r_2 r_3\]
gives a map $d^* : \omega_2 \to \omega_3$. The kernel of this map is $R$
\[0 \to R \overset{\tau}{\longrightarrow} \omega_2 \overset{d^*}{\longrightarrow} \omega_3.\]
The inclusion of $R$ is the canonical map $\tau$. 


\subsubsection{For schemes}


Taking $X = \Spec R$ over $Y = \Spec S$, the question of whether $R$ is an $E$-ring over $S$ is a question about modules on the diagonal
\[\delta : X \to X \times_Y X.\]

For a general map of schemes $f : X \to Y$, the multiplication map
\[\OO_X \otimes_{f^{-1}\OO_Y} \OO_X \to \OO_X\]
gives rise to a canonical element $\tau \in \omega_{X/X \times_Y X}$ in a canonical coherent sheaf on $X$. There is again a universal open inside $X$ on which it is an $E$-ring over $Y$. 




\subsection{Log version}

In log geometry, the diagonal $X \times_Y X$ is different -- fiber products of log schemes are different from those of underlying schemes. It is easier to be a log $E$-ring in general than simply an $E$-ring. Examples include blowups at strata. 







\subsubsection{failed hochschild stuff}

Commented out. 

\begin{comment}

Write $R^{\otimes_S n} = R \otimes_S \cdots \otimes_S R$ for the tensor product of $n$ copies of $R$ over $S$. There is an augmented simplicial object
\[\cdots R^{\otimes_S 3} \rightrightarrows R^{\otimes_S 2} \to R\]
called the bar resolution \cite[\S 8.6.12]{weibel}. Let $C_\bullet$ be the chain complex 
\[C_\bullet : \qquad \cdots R^{\otimes_S 3} \to R^{\otimes_S 2} \to R,\]
where the maps are alternating sums $\sum (-1)^k d_k$ of the face maps $d_k$. That is:
\[C_k = R^{\otimes_S k+2}, \qquad d(r_0 \otimes \cdots \otimes r_{k+1}) \coloneqq \sum_{0 \leq i \leq k} (-1)^i r_0 \otimes \cdots \otimes r_i r_{i+1} \otimes \cdots \otimes r_{k+1}.\]
\Leo{This chain complex is exact! That's the point of the bar resolution!}




This is the bar resolution $B(R, R)$, not the Hochschild complex. The difference is
\[HH(R, N) = B(R, R) \otimes_{R \otimes_S R^{\rm op}} N\] 
if $N$ is a bimodule. Our rings are commutative, $R = R^{\rm op}$. So cohomology of the bar complex may be thought of as Hochschild of the free bimodule
\[HH(R, R \otimes_S R) = B(R, R) \otimes_{R \otimes_S R} R \otimes_S R = B(R, R).\]



Because $C_\bullet$ is a complex, any $f : R \to N$ restricts to a map $\varphi : R \otimes_S R \to N$ that restricts to zero on $R^{\otimes_S 3}$. In order for $\varphi$ to come from a map $f : R \to N$, it must be the case that
\[\varphi|_{R^{\otimes_S 3}} = 0.\]
I.e., $\varphi$ must be a Hochschild cocycle. 
\Leo{This is automatic! The alternating sum map $R\otimes_S R \to R$ is the zero map! I'm off one degree somehow}
\Leo{Not Hochschild! This is the bar construction}


Supposing $\varphi$ is a Hochschild cocycle, it comes from a map $f : R \to N$ if and only if it is a boundary, or its class vanishes in Hochschild cohomology. 

Recall that Hochschild homology $HH_*(R/S, R)$ in degree one is isomorphic to the K\"ahler differentials \cite[Proposition 9.2.2]{weibel}
\[HH_1(R/S, R) \simeq \kah{R/S}\]
and Hochschild cohomology is dually isomorphic to derivations
\[HH^1(R/S, R) \simeq {\rm Der}_S(R, R).\]


We summarize the above observations in an improved answer to Question \ref{q:lenstra}:

\begin{answer}

A map $\varphi : R \otimes_S R \to R$ is the restriction of a map $f : R \to R$ if and only if:
\begin{itemize}
\item 
$\varphi$ is a Hochschild cocycle, i.e., 
\[\varphi(ab, c) - \varphi(a, bc) = \]

\end{itemize}


\end{answer}





\end{comment}





\bibliographystyle{alpha}%Used BibTeX style is unsrt
\bibliography{zbib}


\end{document}
